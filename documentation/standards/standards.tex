%
% ---------- header -----------------------------------------------------------
%
% project       phd
%
% license       GPL
%
% file          /home/mycure/infinit/documentation/standards/standards.tex
%
% created       julien quintard   [wed oct 17 13:30:55 2007]
% updated       julien quintard   [thu mar 13 03:16:40 2008]
%

%
% ---------- packages ---------------------------------------------------------
%

\documentclass[10pt,a4wide]{article}

\usepackage[english]{babel}
\usepackage[T1]{fontenc}
\usepackage{a4wide}
\usepackage{pgf,pgfarrows,pgfnodes,pgfautomata,pgfheaps,pgfshade}
\usepackage{fancyheadings}
\usepackage{multicol}
\usepackage{indentfirst}
\usepackage{graphicx}
\usepackage{color}
\usepackage{xcolor}
\usepackage{verbatim}
\usepackage{aeguill}

%
% ---------- page -------------------------------------------------------------
%

\pagestyle{fancy}
\pagenumbering{roman}

%
% ---------- indentation ------------------------------------------------------
%

\setlength{\footrulewidth}{0.3pt}
\setlength{\parindent}{0.3cm}
\setlength{\parskip}{2ex plus 0.5ex minus 0.2ex}

%
% ---------- verbatim ---------------------------------------------------------
%

\definecolor{verbatimcolor}{rgb}{0,0.4,0}

\makeatletter
\renewcommand{\verbatim@font}
  {\ttfamily\footnotesize\color{verbatimcolor}\selectfont}
\makeatother

%
% ---------- commands ---------------------------------------------------------
%

\newcommand\ie[0]{\textit{i.e}}
\newcommand\eg[0]{\textit{e.g}}
\newcommand\etc[0]{\textit{etc.}}

%
% ---------- headers ----------------------------------------------------------
%

\lhead{\scriptsize{The Infinit File System}}
\rhead{}

\lfoot{}
\rfoot{\scriptsize{Coding Standards}}

%
% ---------- title ------------------------------------------------------------
%

\title
{
  Coding Standards
  \vspace{2cm}
}

\author
{
  \small{Julien Quintard} \\
  \scriptsize{\textit{[firstname]}.\textit{[lastname]}@cl.cam.ac.uk}
  \vspace{1cm}
}

\date{\scriptsize{\today}}

%
% ---------- document ---------------------------------------------------------
%

\begin{document}

\maketitle

\thispagestyle{empty}

%
% ---------- problematic ------------------------------------------------------
%

This short document presents the coding standards of the Infinit file system
and related.

\section{Problematic}

These coding standards were introduced since the \textit{C++} programming
language and more generally most of the object-oriented programming languages
do not provide what the authors expect from them.

The main limitation of such languages concerns the constructors and the fact
that they do not return any value. It is therefore quite complex to indicate
that an error occured in the constructor.

The only solution consists in throwing an exception but authors are against
the idea of using exceptions as they make the source code less readable
and are often misused for indicating errors.

Tricks have been proposed for alleviating this limitation. One consists in
setting an attribute to \textit{true} when the constructor succeeds and
a method or operator to test the success of the object construction.

These tricks have also been rejected by the authors but for a different reason.

Indeed, for performance optimisations, the authors wanted the possibility
to instantiate uninitialised objects so that they can be passed to another
function or method. Then, the object is eventually initialised by the called
function or method.

This is especially very useful when creating local objects \ie{} residing
in the stack.

Besides, the authors believe that when classes provide too many constructors,
those start lacking meaning.

These reasons led the authors to avoid using constructors.

%
% ---------- rules ------------------------------------------------------------
%

\section{Rules}

This section lists the rules established according to the previously presented
problematic:

\begin{enumerate}
  \item
    Every class must provide a \textbf{single} constructor. This constructor
    must not perform any action that could lead to unsuccessful results.

    \-

    This constructor should limit itself to initialising attributes with
    default values;
  \item
    The destructor must both release any reserved resources including
    locks, memory and so forth; and re-initialise the attribute so that
    the object can be re-used;
  \item
    Methods are provided for actually initialising the object. These methods
    take different name according to their usage so that they are meaningful
    compared to using multiples constructors.

    \-

    Especially the following names should be used:

    \begin{itemize}
      \item
	\textit{Create()} for creating the object from nothing more than
	the passed arguments;
      \item
	\textit{Generate()} when the method's result is different when called
	multiple times with the same arguments;
      \item
	\textit{Clone()} for cloning the object \ie{} creating an exact
	copy with the same values as for the copy constructors;
      \item
	\textit{Wrap()} for creating an object which does not re-allocate
	the given arguments but simply wraps them in a new object.
    \end{itemize}
  \item
    In very rare cases, an \textit{Initialise()} method could be found,
    especially in static classes.
\end{enumerate}

\end{document}
