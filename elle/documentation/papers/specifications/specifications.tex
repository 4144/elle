%
% ---------- header -----------------------------------------------------------
%
% project       phd
%
% license       GPL
%
% file          /home/mycure/infi...papers/specifications/specifications.tex
%
% created       julien quintard   [wed oct 17 13:30:55 2007]
% updated       julien quintard   [thu mar 13 03:17:21 2008]
%

%
% ---------- packages ---------------------------------------------------------
%

\documentclass[10pt,a4wide]{article}

\usepackage[english]{babel}
\usepackage[T1]{fontenc}
\usepackage{a4wide}
\usepackage{pgf,pgfarrows,pgfnodes,pgfautomata,pgfheaps,pgfshade}
\usepackage{fancyheadings}
\usepackage{multicol}
\usepackage{indentfirst}
\usepackage{graphicx}
\usepackage{color}
\usepackage{xcolor}
\usepackage{verbatim}
\usepackage{aeguill}

%
% ---------- page -------------------------------------------------------------
%

\pagestyle{fancy}
\pagenumbering{roman}

%
% ---------- indentation ------------------------------------------------------
%

\setlength{\footrulewidth}{0.3pt}
\setlength{\parindent}{0.3cm}
\setlength{\parskip}{2ex plus 0.5ex minus 0.2ex}

%
% ---------- verbatim ---------------------------------------------------------
%

\definecolor{verbatimcolor}{rgb}{0,0.4,0}

\makeatletter

\renewcommand{\verbatim@font}
  {\ttfamily\footnotesize\selectfont}

\def\verbatim@processline
  {\hskip15ex{\color{verbatimcolor}\the\verbatim@line}\par}

\makeatother

%
% ---------- commands ---------------------------------------------------------
%

\newcommand\ie[0]{\textit{i.e}}
\newcommand\eg[0]{\textit{e.g}}
\newcommand\etc[0]{\textit{etc.}}

%
% ---------- headers ----------------------------------------------------------
%

\lhead{\scriptsize{The Infinit File System}}
\rhead{}

\lfoot{}
\rfoot{\scriptsize{The Elle Library Specifications}}

%
% ---------- title ------------------------------------------------------------
%

\title
{
  The Elle Library Specifications
  \vspace{2cm}
}

\author
{
  \small{Julien Quintard, Nicolas Grandemange} \\
  \scriptsize{\textit{[firstname]}.\textit{[lastname]}@cl.cam.ac.uk}
  \vspace{1cm}
}

\date{\scriptsize{\today}}

%
% ---------- document ---------------------------------------------------------
%

\begin{document}

\maketitle

%
% ---------- introduction -----------------------------------------------------
%

\section{Introduction}

The Elle library provides functionalites for performing operations which
are very common in distributed file systems. Such functionalities include
cryptography, serialisation, communication, asynchronous behaviours and so
forth.

The remaining of the paper presents the specifications of the Elle library
by giving examples of what programmers expect when using such a library.

%
% ---------- packages ---------------------------------------------------------
%

\section{Packages}

The library is divided into packages, each of which providing an interface
for manipulating a specific feature from communication to cryptography.

%
% misc
%

\subsection{\textit{misc}}

The \textit{misc} package deals with concepts common to two or more other
packages. Especially, this package provides classes for managing memory
in an efficient way, trying to reduce allocation, copy \etc{}

Besides managing memory, the \textit{misc} package tries to simplify types
management. Indeed, the C++ programming language embeds all of the legacy
types brought by the C language.

Finally, the package provides an error reporting system which enables
applications relying on the Elle library to get back a history of messages
when an error occurs.

Elle types include \texttt{Byte}, \texttt{Integer} and \texttt{Quad} for
1-byte, 4-byte and 8-byte integers. \texttt{Double} is similar to the
\texttt{double} type in C while \textit{Char} represents a character.

The \texttt{Region} class represents a region of memory. The \texttt{Chunk}
class is a specialisation of a region since chunks are fixed size memory
regions. On the other hand, \texttt{Buffer} objects can grow in size by
appending contents.

\texttt{Bucket}s also represent dynamic memory region although the dynamicity
is expressed through a list or regions. Such buckets are very useful for
building virtually aggregated buffers. Indeed, in many cases --- especially
with cryptographic functions --- information is processed on a block basis.
It is therefore, most of the time, needless to aggregate multiple blocks
into a physically continuous memory chunk, avoiding multiple re-allocations
and memory copies.

%
% crypto
%

\subsection{\textit{crypto}}

The \textit{crypto} package is likely to be used by the application directly
but is also used by other packages including the \textit{channel} package.

The \textit{crypto} package provides functions for encrypting, signing,
hashing \etc{} through different cryptosystems. The package also provides
ways for importing and exporting keys complying with \textit{OpenSSH}, enabling
users to re-use their keys.

Cryptographic types \texttt{Input}, \texttt{Output} and their specialisations
\texttt{Plain}, \texttt{Code}, \texttt{Clear}, \texttt{Digest} and
\texttt{Signature} directly rely on the \textit{misc} types \texttt{Region}
\texttt{Bucket} \etc{}

%
% archive
%

\subsection{\textit{archive}}

Serialisation is often referred as the process consisting in transforming
a class instance \ie{} object into a raw format enabling it to be transmitted
or even stored.

The \textit{archive} package enables class developers to complete their classes
with an operator for easily building archives \ie{} serialised objects. In
addition, the package provides everything necessary for performing the
inverse operation \ie{} extracting an archive for re-building an active
object.

The provide technique allows developers to decide which parts of classes
to serialise while languages like \textit{Java} automatically serialise
the whole object.

%
% channel
%

\subsection{\textit{channel}}

The most common operation nodes of a distributed system perform is
communicating. The Elle library hence has to provide primitives for
very easily allowing processes to communicate in various modes: authenticated,
encrypted, \textit{FIFO} \etc{}

The Infinit file system is implemented through multiple local processes.
These processes need to communicate but also need to be authenticated so
that one process knows that the other is properly running under the
identity of the user it pretends to act on the behalf. The same goes for
nodes running in the distributed system, hence likely to be on a different
continent.

\subsubsection{\textit{Resume}}
Synchronus
\begin{verbatim}		   	   
Channel("42.42.51.69",    4242,     SYNC);          ==>  TCP v4
Channel(85654, SYNC);                               ==>  FIFO IPC
\end{verbatim}
Asynchronus
\begin{verbatim}		   	   
Channel("42.42.51.69",    4242,     ASYNC);         ==>  UDP v4
\end{verbatim}


\subsubsection{\textit{Synchronus channel over FIFO IPC}}

Code exemple:

Header:
\begin{verbatim}
#include <elle/channel/Channel.hh> //Channel Class
#include <elle/misc/Report.hh>     //Error manager
#include <iostream>
#include <string>

using namespace elle;
using namespace elle::misc;
using namespace elle::channel;
\end{verbatim}

Process code:
\begin{verbatim}
int		prog1(int pid)     // pid ==> The pid of the remote process
{
  Channel	*chan;
  std::string	msg;
  
  chan = new Channel(pid, SYNC);   //Constructor for synchronus FIFO IPC
  
  if (chan->connect() != StatusOk)
    report.Display("I: ");         //Display debuging informations 
                                   // "I: " represent the prefix you want to print
                                   // on stderr
  
  if (chan->recive(msg) != StatusOk)
    report.Display("I: ");         //Display debuging informations

  std::cerr << "I: recive: " << msg << std::endl;
  
  if (chan->send("Chiche donne tout\n") != StatusOk)
    report.Display("I: ");
  
  delete chan;
  return (0);
}
\end{verbatim}



\end{document}
